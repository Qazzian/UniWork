\setcounter{slide}{86}

%
% slide 87
%
\begin{slide}{}
{\bf Static Checking}\\
(also known as ``context checking'' or ``semantic analysis'')

\begin{itemize}
\item type checking
\item flow of control checks
\item uniqueness checks
\item name related checks
\end{itemize}

\end{slide}
%
% slide 88
%
\begin{slide}{}
{\bf Static Checking by Attribute Evaluation}

An attribute grammar describes static checks.

Attributes used in static checking are called
{\em semantic attributes}.  

The abstract syntax tree (AST) is traversed
and decorated with attributes until it
contains enough information so that static
checks can be made.

\end{slide}
%
% slide 89
%
\begin{slide}{}
{\bf Static Checking}

Static checking can be performed after the AST
has been constructed, or ``on-the-fly'', while
it is being built.

In the first case, the AST is traversed, usually
more than once, and attributes are evaluated each
time a node is visited.

On-the-fly evaluation methods operate in conjunction
with the parser (just like the tree building routines).
They are used for single pass compilation.

The parsing method used limits the kind of attribute
flow that can be accommodated when on-the-fly
evaluation is carried out.
\end{slide}

%
% slide 90
%
\begin{slide}{}
{\bf Intermediate Representations}

The ``front end'' or analysis phases of a compiler
produce intermediate code, which is processed
by the code generator or ``back end'' of the compiler.

Three kinds of intermediate representation are
\begin{itemize}
\item abstract syntax trees,
\item postfix notation and
\item 3-address code.
\end{itemize}
\end{slide}
%
% slide 91
%
\begin{slide}{}
{\bf 3-Address Code}

3-Address Code is like assembly language.  It consists of a
sequence of statements like
\begin{verbatim}

    x := y op z

\end{verbatim}
where x, y and z are names, constants, or temporary names generated
by the compiler and op is an operator.

Statements can have labels, and there are statements that modify
flow of control.  For example:  
\begin{verbatim}
    label: x := y op z
    goto label  
    if x relop y goto label
\end{verbatim}
\end{slide}
%
% slide 92
%
\begin{slide}{}
{\bf 3-Address Code}

Example: 3-address code corresponding to the syntax
tree shown earlier.
\begin{verbatim}

    t1 := -c
    t2 := b * t1
    t3 := -c
    t4 := b * t3
    t5 := t2 + t4
    a  := t5

\end{verbatim}
\end{slide}
%
% slide 93
%
\begin{slide}{}
{\bf Translation into 3-Address Code}

Syntax directed translation of the source program into
3-address code can be defined using an attribute grammar.

Example: An attributed grammar rule for translating a
`while' loop to 3-address statements:               
\begin{verbatim}
  S ::= while E do S1
    {S.begin := newlabel;
     S.after := newlabel;
     S.code  :=
       S.begin <> `:' <> E.code <>
       `if' <> E.place <> `= 0' <>
       `goto' <> S.after <>
       S.code <>
       `goto' S.begin  <>
       S.after `:'
    }
\end{verbatim}
\end{slide}
%
% slide 94
%
\begin{slide}{}
{\bf 3-Address Statements}
{\small
\begin{itemize}
\item Basic Assignments\\
    x := y op z\\
    x := op y\\
    x := y
\item Transfer of Control\\
    goto label\\
    if x relop y goto label
\item Procedure Call\\
    param x1\\
    param x2\\
    ...\\
    param xn\\
    call p,n\\
\item -- and Return\\
    return y
\item Indexing\\
    x[i] := y\\
    y := y[i]
\item Operations involving pointers\\
    x := \&y\\
    x := *y\\
    *x := y
\end{itemize}
}
\end{slide}
%
% slide 95
%
\begin{slide}{}
{\bf 3-address code for computing a scalar product -- an
example from Aho, Sethi and Ullman}

Assume a and b are two vectors of length 20. The
following 3 address statements compute the scalar
product of a and b. (It assumes that the target machine
has four byte words).

\begin{verbatim}

 (1)  prod := 0
 (2)  i := 0
 (3)  t1 := 4 * i
 (4)  t3 := a[t1]
 (5)  t4 := b[t1]
 (6)  t5 := t3 * t4
 (7)  prod := prod + t5
 (8)  i := i + 1
 (9)  if i <= 20 goto (3)
\end{verbatim}

\end{slide}

