\setcounter{slide}{42}
%
% slide 43
%
\begin{slide}{}
{\bf Unix Pattern Matching}
\begin{itemize}

\item There are inconsistencies between the ways different characters are
used in different Unix applications and situations.

\item In particular, characters are treated differently by the shell 
in filename expansion and by applications for pattern matching.

\item For example, * means match 0 or more of the preceding expression 
in vi, gvim, sed, awk, grep and egrep, but * means match any string of
characters in filename expansion

\end{itemize}
\end{slide}

%
% slide 44
%
\begin{slide}{}
{\bf Unix Pattern Matching}
\begin{itemize}

\item You can use single quotes (or, in some cases, double quotes) to
bypass the shell and pass special characters to an application.

\item Look up `Unix in a Nutshell', section 6, Pattern Matching, for
more detail

\item Alternatively, look up the man pages for the applications you are
using (man vi will tell you all you need for gvim).

\end{itemize}

\end{slide}

%
% slide 44
%
\begin{slide}{}
{\bf perl - Practical Extraction and Report Language}

\begin{itemize}
\item scanning text files
\item extracting information
\item printing reports
\item features of C, sed, awk and sh
\item When sed, awk and shell scripts are not quite enough \ldots
\item \ldots and C is a bit too much
\item \ldots perl is likely to be just right.
\end{itemize}

\end{slide}

%
% slide 45
%
\begin{slide}{}
{\small
{\bf perl example -- derived from examples by Lynda Thomas}

\begin{verbatim}

manuel% more read_a_file.pl
#!/usr/local/bin/perl

print "Filename? ";
$file = <STDIN>;        # filename is on standard input
open(FILE,$file);       # open the file - FILE is a filehandle
@file_content = <FILE>; # read into an array
close(FILE);
print @file_content;    # print the array

\end{verbatim}
}
\end{slide}

%
% slide 46
%
\begin{slide}{}
{\bf perl example}

\begin{verbatim}
manuel% cat names
lynda dave tom rory lynda rory lynda

manuel% read_a_file.pl
Filename? names
lynda dave tom rory lynda rory lynda

manuel% ls -l read_a_file.pl
-rwxr-----   1 eds      staff        
    264 Oct 13 14:33 read_a_file.pl

\end{verbatim}
\end{slide}

%
% slide 47
%
\begin{slide}{}
{\small
{\bf perl identifiers and types}

\begin{verbatim}
manuel% cat trythis.pl
#!/usr/local/bin/perl

$scalar_string_id = "this is a string";  
print $scalar_string_id . "\n";

$scalar_number_id = 7;
print "scalar_number_id = " . $scalar_number_id . "\n";

@array_id = ("peach","fig","apricot");
print @array_id[0] .  @array_id[2] . @array_id[1] . "\n\n";

%hash_id = ("peach", 2, "fig", 8, "apricot", 3);
print "We have " . @hash_id{"peach"} . " peaches\n";
print "We have " . @hash_id{"fig"} . " figs\n";
print "We have " . @hash_id{"apricot"} . " apricots\n";

\end{verbatim}
}
\end{slide}

%
% slide 48
%
\begin{slide}{}
{\bf perl identifiers and types}

\begin{verbatim}

manuel% trythis.pl
this is a string
scalar_number_id = 7
peachapricotfig

We have 2 peaches
We have 8 figs
We have 3 apricots

\end{verbatim}
\end{slide}

%
% slide 49
%
\begin{slide}{}
{\bf perl loops -- from an example by Mike Slattery and Lynda Thomas}

\begin{verbatim}
manuel% cat loopy.pl
#!/usr/local/bin/perl

while (<>) {
  @words = split;
  foreach $w (@words) {
    $count{$w}++;
  }
}

@sortedkeys = sort keys(%count);

foreach $w (@sortedkeys) {
  print "$w\t$count{$w}\n";
}

\end{verbatim}

\end{slide}

%
% slide 50
%
\begin{slide}{}
{\bf perl loops}

\begin{verbatim}

manuel% cat names
lynda dave tom rory lynda rory lynda

manuel% loopy.pl names
dave    1
lynda   3
rory    2
tom     1

\end{verbatim}
\end{slide}

%
% slide 50
%
\begin{slide}{}
{\bf Patterns}
\begin{verbatim}
manuel% cat pattern.pl
#!/usr/local/bin/perl

while (<>) {
  @words = split;
  foreach $w (@words) {
    $count{$w}++;
  }
}

@sortedkeys = sort keys(%count);

foreach $w (@sortedkeys) {
  print "$w\t$count{$w}\n" if ($w =~ m/^l/);
}
\end{verbatim}
\end{slide}

%
% slide 51
%
\begin{slide}{}
{\bf Patterns}
\begin{verbatim}
manuel% cat names 
lynda dave tom rory lynda rory lynda marilyn

manuel% pattern.pl < names
lynda   3

\end{verbatim}
\end{slide}

%
% slide 54
%
\begin{slide}{}
{\small
{\bf perl - a subroutine}

\begin{verbatim}
manuel% cat subroutine.pl
#!/usr/local/bin/perl

# from a program by Mike Slattery, June 1996

# swap takes a string, and replaces any words in %table by 
# corresponding replacements

%table = ('i','you', 'you','i', 'my','your', 'your','my');

sub swap {
  local ($in) = @_;
  local ($w, $head, $tail, $new);

  if ($in =~/[a-z]+/) {
    $w = $&;    # the match
    $head = $`; # before the match
    $tail = $'; # after the match

    # look up $new in the table; if not found, $new = $w
    $new = $table{$w} || $w;

    # put the sentence back together
    $head.$new.&swap($tail); 
  }

  else { $in }
}
\end{verbatim}
}
\end{slide}
%
% slide 55
%
\begin{slide}{}
{\bf \ldots and the main program}

\begin{verbatim}

while ($input = <>) {
   chop $input;
   $input =~ tr/A-Z/a-z/;
   print &swap($input)."\n";
}

\end{verbatim}
\end{slide}

%
% slide 57
%
\begin{slide}{}

{\bf Running the program}

\begin{verbatim}
manuel% subroutine.pl
i hate computers
you hate computers
i like running
you like running
you are daft
i are daft
i'm daft too
you'm daft too
i love computers really
you love computers really
^D
\end{verbatim}
\end{slide}


%
% slide 58
%
\begin{slide}{}
{\bf perl - some url's}

\begin{itemize}
\item http://language.perl.com/
\item http://www.perl.com/pace/pub
\item http://www.perl.org/
\end{itemize}

\end{slide}




