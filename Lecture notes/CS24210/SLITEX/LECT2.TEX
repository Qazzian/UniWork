\setcounter{slide}{22}
%
% slide 12
%
\begin{slide}{}
{\bf Regular Expressions}

Regular expressions provide a notation for describing
the tokens used in modern programming languages.

A regular expression specifies a pattern.  The set of
character strings that match a regular expression is
called a regular set.

For example,

    $ (underscore|letter)(underscore|letter|digit)* $

is a regular expression that specifies the set of
Java identifiers -- provided \ldots

\end{slide}
%
% slide 13
%
\begin{slide}{}
{\small
{\bf Regular expressions and what they match}

\begin{tabular}{|p{4.5cm}|p{3cm}|p{6cm}|}
\hline
Conventional notation & Unix notation & matches \\
\hline
\hline
$\phi$ & & no string \\
&&\\
\hline
$\epsilon$ & & the empty string \\
&&\\
\hline
$s$ & $s$ & the string $s$ \\
&&\\
\hline
$A|B$ & $[AB]$ & any string that matches $A$ or $B$ \\
&&\\
\hline
$AB$ & $AB$ & any string that can be split into a part that matches $A$ followed by a part that matches $B$\\
&&\\
\hline
$A*$ & $A*$& any string that consists of zero or more $A$'s \\
&&\\
\hline
\end{tabular}

$s$ is any character string\\
$A$ and $B$ are regular expressions
}
\end{slide}

%
% slide 14
%
\begin{slide}{}
{\bf Shorthands}
\begin{itemize}
\item $ A+ $ one or more A's\\
equivalent to $ AA* $
\item $ A? $ zero or one A \\
equivalent to $[\epsilon A]$
\item $ [0-9] = [0123456789] $
\end{itemize}
\end{slide}
%
% slide 15
%
\begin{slide}{}
{\bf Using Regular Expressions}
\begin{itemize}
\item Unix applications 
\begin{itemize}
\item substitution using gvim\\{\small $<range>s/<regexp>/<substitution>$}
\item searching files for patterns\\$grep~<regexp>~<path>$
\end{itemize}
\item Specifying valid input \ldots
\vfill
\item perl pattern specifications\\
 $t.e$ matches $the$ or $tee$ but not $tale$
\end{itemize}
\end{slide}
%
% slide 16
%
\begin{slide}{}
{\bf Constructing regular expressions}

Write a regular expression that matches the Java keyword {\bf class}
\vfill
\end{slide}
%
% slide 17
%
\begin{slide}{}
{\bf Constructing regular expressions}

Using conventional notation, extend your regular expression so that it
matches any of the Java keywords: {\bf abstract} {\bf class} {\bf extends}

\vspace{10ex}

Write an equivalent regular expression using Unix notation.

\vspace{10ex}
\end{slide}
%
% slide 18
%
\begin{slide}{}
{\bf Constructing regular expressions}

Using conventional notation,
write a regular expression that matches unsigned integers

\vspace{7ex}

Do the same, this time using Unix notation.

\vspace{7ex}
\end{slide}
%
% slide 18
%
\begin{slide}{}
{\bf Constructing regular expressions}

Extend each of your previous regular expressions so that they
match signed integers.

\vfill
\end{slide}
%
% slide 15
%
\begin{slide}{}
{\bf More information}

To find out all about regular expressions in Unix, see
\begin{verbatim}

man -s 5 regexp
\end{verbatim}
\end{slide}

%
% slide 15
%
\begin{slide}{}
{\bf Finite State Automata}

A finite state automaton (FSA) recognises strings
that belong to a regular set.

It consists of
\begin{itemize}
\item a finite set of states,
\item a set of transitions from state to state,
\item a start state and
\item a set of final states, called ``accept states''.
\end{itemize}

A FSA can be represented by a transition diagram -- a
directed graph whose nodes are labelled with state symbols,
and whose edges are labelled with characters.
\end{slide}
%
% slide 16
%
\begin{slide}{}
{\bf FSA that recognises numbers}

\vspace{3ex}
\epsfbox{FSAnum.eps}

This FSA is deterministic -- for a given state and
input character, the next state is uniquely determined.
\end{slide}
%
% slide 17
%
\begin{slide}{}
{\bf Fragment FSA for a typical programming language}

\vspace{3ex}
\epsfbox{FSAfragment.eps}

\end{slide}
%
% slide 18
%
\begin{slide}{}
{\bf Pseudo code for a scanner}

Assume one character, `ch', has been read by `getch', which
reads a character, skips white space and produces a listing.

\begin{verbatim}
    case ch of:
        `;' : token := semi
              getch
        `:' : getch; if ch = `='
                        then token := assign
                             getch
                        else token := colon
        `0' .. `9' : token := number
                     while ch in [`0' .. `9']
                           getch
                     endwhile
        ... &c.
\end{verbatim}

\end{slide}
%
% slide 18
%

\begin{slide}{}
{\bf Transition Tables}

In general, a deterministic FSA can be programmed as a language
independent driver that interprets a transition table.
The transition table for the FSA that recognises numbers
looks like this.

\begin{tabular}{|l||c|c|c|c|c|c|c|}\hline
& \multicolumn{7}{c|}{Character}\\ \cline{2-8}
State & + & - & . & e & E & 0 \ldots 9 & other \\
\hline
0 & 1 & 1 & & & & 2 & \\
1 & & & & & & 2 & \\
2 & & & 4 & 5 & 5 & 2 & 3\\
3 & & & & & & & \\
4 & & & & 5 & 5 & 4 & 8 \\
5 & 6 & 6 & & & & 7 & \\
6 & & & & & & 7 & \\
7 & & & & & & 7 & 8 \\
8 & & & & & & & \\
\hline
\end{tabular}

\end{slide}
%
% slide 19
%
\begin{slide}{}
{\bf A scanner driver}

\begin{verbatim}

    State:= InitialState;
    Read(CurrentChar);
    loop
        NextState := TT(State,CurrentChar);
        exit when NextState = Error or
                  CurrentChar = EOF;
        State := NextState;
        Read(CurrentChar);
    end loop
    if State in FinalStates
        then return valid token
        else lexical error
    end if;

\end{verbatim}
\end{slide}
%
% slide 20
%
\begin{slide}{}
{\small
{\bf Using a Scanner Generator}

There are algorithms for translating regular expressions 
to nondeterministic
finite state automata (NFA), and for translating NFA 
to deterministic finite state automata (DFA).

These are used in scanner generators like lex, developed by
M.E. Lesk and E. Schmidt of Bell labs.

\vspace{3ex}
\epsfbox{ScannerGen.eps}

Programming effort is limited to describing
lexemes -- an example of declarative programming.

Given input consisting of character class definitions,
regular expressions with corresponding actions, and
subroutines, 
lex generates a scanner called yylex.
}
\end{slide}

%
% slide 21
%
\begin{slide}{}
{\bf Symbol Tables}

The compiler uses symbol tables to collect and use information
about names that appear in a source program.

Each time the scanner encounters a new name, it makes a symbol
table entry for the name.  Further information is added during
syntactic analysis, and this information is used (and further
extended) during semantic analysis and code generation.

\end{slide}
%
% slide 22
%
\begin{slide}{}
{\small
{\bf Symbol Tables}

The kind of information stored in a symbol table entry includes items like
\begin{itemize}
\item the string of characters denoting the name,
\item the block or procedure in which the name occurs,
\item attributes of the name (e.g. type and scope),
\item the use to which the name is put (e.g. formal parameter, label \ldots)
\item parameters, such as the number of dimensions in an array and their
      upper and lower limits, and
\item the position in storage to be allocated to the name (perhaps).
\end{itemize}
}

\end{slide}
%
% slide 23
%
\begin{slide}{}
{\small
{\bf The Symbol Table as an Abstract Data Type}

Conceptually, a symbol table is a collection of pairs:
\begin{quote}
    (name,information)
\end{quote}
with operations to
\begin{itemize}
\item determine whether or not a name is in the table,
\item add a new name,
\item access the information associated with a given name,
\item add new information for a given name, and
\item delete a name, or a group of names.
\end{itemize}
}
\end{slide}
%
% slide 24
%
\begin{slide}{}
{\bf Symbol Table Implementation}

Symbol tables are typically implemented using
\begin{itemize}
\item linear lists,
\item hash tables, or
\item various tree structures.
\end{itemize}

In selecting an implementation, speed of access is traded
against structural complexity.

\end{slide}


