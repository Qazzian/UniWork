\documentstyle[ucw,epsf]{article}
\begin{document}



\section{Abstraction\footnote{This section is quoted from Mark Ratcliffe, CS12220/CS12320 lecture notes, Introduction, available via http://www.aber.ac.uk/~\~dcswww/Dept/Teaching/Courses/CS12220/}}

`People typically understand the world by constructing mental models of portions of it. We
try to understand things so that we can interact with them; a mental model is a simplified
view of how something works so that we can interact with it.
 
`Consider driving a car - we abstract out unnecessary detail. There will of course be
different models of the same thing for different purposes. A driver's model of a car will be
different to that of a motor mechanic.
 
`Consider maps - a road map (1:50,000) models how best to drive from one location to
another, a topographical map (at a scale of 1:10,000) models the contours of the landscape,
perhaps to plan a system of hiking trails. Each is useful for its specific purpose because of
what it omits. 
 
`Abstraction is essential to the human mind, and is an immensely powerful tool for
dealing with complexity. Consider, for example, the mental feat involved in memorising
numbers. Perhaps seven digits are all that we can manage. But if you group them and call
them a telephone number, you have relegated the individual digits to a lower level. You
have created a higher, abstract level at which all seven digits are a single entity. Using this
mechanism, you can now memorize perhaps seven telephone numbers, thus increasing
your ability to deal with complexity by an order of magnitude. Grouping several conceptual
entities into one is a powerful mechanism in the service of abstraction.
 
\end{document}
