
\documentstyle[ucw,11pt]{article}
\pagestyle{empty}
\title{CS24210: Intelligent Editing}
\author{Edel Sherratt}
\date{\today}
\begin{document}
\maketitle

This practical is about editing a set of Stack programs, firstly to
improve them and secondly to change them into a similar set of Queue
programs.  To do this, you will use {\tt gvim}, a screen
editor, and {\tt sed}, a stream editor.  

{\tt gvim} has two modes --
an insert mode and a command mode.  Commands can be issued directly,
or, usually in the case of more elaborate commands, by
typing {\tt :} followed by the rest of the command. It is this las
kind of command that you will be using. 

Commands in both editors look rather similar. For example, here
is the general form of the substitute command, which you will be using.

$<range>s/<pattern>/<replacement>/<flags>$


$<range>$ says which lines are to be modified, $s$ is the substitute 
command itself,
and $<pattern>$ and $<replacement>$ indicate the changes to be made.
$\$$ is a special symbol, which means the last line in the file, when
it appears in a $<range>$, and which means the end of line when it
appears in a $<pattern>$.  The flag you'll mainly use will be $g$, which
says do the replacement throughout each line visited, and not just once.

If you happen to get into insert mode while using {\tt gvim}, you can get 
back to command mode by typing $escape$. You can undo the most recent
command by typing {\tt u}.  And, if all else fails, you can always restart
the practical.

Now -- have fun.

\begin{enumerate}

\item Download a copy of EasyStack.tar from the CS24210 web page. Unpack it
using {\tt tar xvf EasyStack.tar}.  This will create a subdirectory called
EasyStack in your current working directory.  The subdirectory will contain
Stack.java, Fruit.java and five test harnesses --  StackHarness1.java, \ldots StackHarness5.java.

\item Change directory to EasyStack: {\tt cd EasyStack}.  Your next
task is to improve the identifiers in Stack.java using the {\tt gvim}
editor: {\tt gvim Stack.java}.  Here are some {\tt gvim} commands to do this:

\begin{verbatim}

:1,$s/I\>/Interface/g
:1,$s/D\>/DataType/g
:1,$s/V\>/Visitor/g

:1,$s/\<s\>/stack/g
:1,$s/\<_s\>/_stack/g

:1,$s/\<f\>/thing/g
:1,$s/\<_f\>/_thing/g

\end{verbatim}

Note the use of $\\<$ and 
$\\>$ to mark the beginnings and
ends of words.

Remember, if you have trouble working out what is matched by 
the patterns used in any of these commands, just check out 
{\tt man -s 5 regexp}

Use {\tt :x} to exit {\tt gvim} -- or use the {\tt save-quit} option on
the {\tt file} pull down menu.

Now improve the identifiers in each of the StackHarness files in the
same way.  Use {\tt gvim StackHarness[1-5].java} to edit each of these
files in turn.  Use the following two substitutions. 
\begin{verbatim}

:1,$s/D\>/DataType/g
:1,$s/V\>/Visitor/g

\end{verbatim}

When you've finished editing StackHarness1.java, use
{\tt :w} to save it and {\tt :n} to move on to the next file.  Don't use
the {\tt save-quit} option this time! 

When you've finished editing StackHarness.java, use {\tt :x} to exit.

\item Compile and run Stack.java and Fruit.java with each of the five
test harnesses in turn.
\begin{verbatim}
javac Stack.java
javac Fruit.java
javac StackHarness[1-5].java

java StackHarness1

...

java StackHarness5
\end{verbatim}

\item Create a new directory, called EasyQueue {\tt mkdir EasyQueue} 
and copy all the .java
files from EasyStack into EasyQueue.  
\begin{verbatim}
cp *.java EasyQueue
\end{verbatim}

In the EasyQueue directory -- {\tt cd EasyQueue} -- rename
Stack.java to Queue.java: {\tt mv Stack.java Queue.java}

\item Your next task is to modify Queue.java so that it really implements
a queue data structure. Here are some gvim commands to do this:
\begin{verbatim}

:1,$s/Stack/Queue/g
:1,$s/Push/Join/g
:1,$s/Pop/Leave/g

:1,$s/stack/queue/g

\end{verbatim}

\item Now for the clever bit!  Line 43 in the file is in the class PopV,
and it now says (following the above substitutions):

\begin{verbatim}

    return (QueueDataType) queue;

\end{verbatim}

See if you can figure out what this gvim command will do, assuming
you execute it as a {\bf single line} without any line break:

\begin{verbatim}

:43s/return (QueueDataType) queue/if (queue instanceof Empty) &;
 else return new Join(thing,(QueueDataType)queue.accept(this))

\end{verbatim}

Try it, but remember, no line break.

You may wish to improve the layout of the resulting code -- please
do this, then save and quit (or use {\tt :x}).

\item It would be possible, but slightly tedious, to use gvim to
edit all the StackHarness files.  Instead, you will use sed to do
this.

Create a file called {\tt sed\_commands} containing the following {\tt sed} commands:
\begin{verbatim}
1,$s/Stack/Queue/g
1,$s/Push/Join/g
1,$s/Pop/Leave/g

1,$s/stack/queue/g

\end{verbatim}

Create a file containing the {\tt bash} script:
\begin{verbatim}
#!bash

let i=1
while [ $i -le 5 ]
do
  sed -f sed_commands StackHarness$i.java > QueueHarness$i.java
  let i=i+1
done
\end{verbatim}

Be very careful to type spaces etc. exactly as given above.

Now use the the command 
\begin{verbatim} 
source $<bash script file>$ 
\end{verbatim}

to modify your StackHarnesses.

\item Finally compile and run your Queue programs:
\begin{verbatim}

javac Queue.java
javac Fruit.java
javac QueueHarness[1-5].java

java QueueHarness1

...

java QueueHarness5
\end{verbatim}

\end{enumerate}


\end{document}
