
%
% slide 1
%
\begin{slide}{}
\begin{center}
{\large \bf CS24210 Syntax Analysis and Topics in Programming Languages}\\
\vspace{3ex}
{\small Edel Sherratt}
\end{center}

{\bf About this Module}

Your Aim: to become familiar with the concepts required for
specifying and implementing  programming lanuguages

Why?

\end{slide}
%
% slide 2
%
\begin{slide}{}
{\bf Learning Outcomes -- for the whole course}

On successful completion of this course, you should be able to
\begin{itemize}
\item make effective use of compilers and other language processing software; 
\item apply the techniques and algorithms used in compilation to other areas 
      of software engineering;
\item understand the need for implementation independent semantics of
     programming languages; 
\end{itemize}
\end{slide}
%
% slide 3
%
\begin{slide}{}
{\bf How the course will work}

\begin{itemize}
\item lectures, will require input from you
\item scheduled practicals 
\item reading -- web based materials and library 
\item own practical work 
\end{itemize}
\end{slide}
%
% slide 4
%
\begin{slide}{}
{\small
{\bf About the Course}

\begin{itemize}
\item Syllabus -- \\
 www.aber.ac.uk/\~~dcswww/Dept/Teaching/\\Syllabus/1999-00/
\item The order will change
\item The content will vary from year to year
\item The examination will be based on the course {\em as taught}
\item For the highest marks,  evidence of additional reading, and 
more specially, {\em coherent thought}  about the taught material
will be demanded
\item This means that your work should be paced
\item Please point out any problems with the course content or
presentation quickly -- during the lecture, if necessary
\end{itemize}
}
\end{slide}

%
% slide 5
%
\begin{slide}{}
{\small
{\bf Useful sources of information}
\begin{itemize}
\item www.aber.ac.uk/\~~dcswww/Dept/Teaching/\\Courses/CS24210
\item N.B. Bennett is not available - so please ignore the statement
in the syllabus that it is essential to purchase that book
\item Alternatives should be in the bookshops during the coming week
\item Aho, Sethi and Ullman, {\em Compilers: Principles, Techniques and Tools,} ISBN~0-201-10194-7
\item Fischer, C.N. and Leblanc, R.J. {\em Crafting a compiler with C,}
ISBN~0-8053-2166-7 (this will be probably be changed to the Java edition 
for next year's course -- which will affect its resale value)
\item Wirth, {\em Compiler Construction}, ISBN~0-201-40353-6 
\end{itemize}
}
\end{slide}


%
% Overview of Language Processing Software - 1 lecture
%
% Learning outcomes - following this lecture, you should
% 
% be aware of the variety and scope of language processing software that you
% have already used or are likely to want to use during your course/career
%
% know why you need to balance practice with transferable knowledge and skills
% 

%
% slide 5
%
\begin{slide}{}
{\bf Language Processing Software}
\begin{itemize}
\item Editors -- e.g. nedit, gvim, vi, sed, \ldots
%\vspace{10ex}
\item Compilers and Interpreters -- e.g. javac, gcc, perl, bash, csh \ldots
%\vspace{10ex}
\item Text formatters -- e.g. \LaTeX, nroff \ldots
%\vspace{10ex}
\end{itemize}
\end{slide}
%
% slide 6
%
\begin{slide}{}
{\small
{\bf \ldots and the Languages they process}
\begin{itemize}
\item gvim: ascii text, programming languages, markup languages
\item javac: java programs
\item gcc: ANSI C programs
\item LaTeX: text marked up with latex formatting instructions
\item html: text marked up with html formatting instructions
\item perl: perl scripts
\item bash: bash shell commands
\item csh: C shell commands
\end{itemize}
}
\end{slide}
%
% slide 7
%
\begin{slide}{}
{\bf How do we learn to use Language Processing Software Effectively?}
\begin{itemize}
\item Learn all of these languages and software systems individually?
\item In a 10-credit module -- \\ 80 hours of your time???!
\item How do we cope with new languages and tools?
\item How do we make sure our knowledge is good for life?
\end{itemize}
\end{slide}
%
% slide 8
%
\begin{slide}{}
{\bf Lasting, transferable knowledge and skills in Language Processing}

\begin{itemize}
\item We need some practice 
\item and also some abstract knowledge -- transferable skills -- 
\end{itemize}

Recall CS12220/12320 -- abstraction as a tool for dealing with complexity.
Read this before the next lecture. Think hard about what it means for
the study of language.
\end{slide}
%\setcounter{slide}{9}
%
% lect 2
%
% Objectives: students should be able to
% analyse the meaning of abstraction with regard to the study of language
% name and describe three major aspects of language study: 
%   viz. syntax, semantics and pragmatics
% state what is meant by a lexeme
% present examples of correct and incorrect Java lexical syntax 
% write regular expressions using the notation described in the Unix chapter 5 man page
%
% slide 9
%
\begin{slide}{}
{\small
{\bf Abstraction as a tool to support effective use of language processing software}

\begin{itemize}
\item What kind of information do we need to keep in mind?\\
  depends on what we're thinking about
\begin{itemize}
\item all languages: the fact that languages have structure; 
\item Java: class structure;
\item perl: pattern structure, loop structure, \ldots
\end{itemize}
\item What kind of information do we omit?
\begin{itemize}
\item all languages: possible structures 
\item Java: semicolons etc.
\item perl: * / $<$ $>$ . etc.
\end{itemize}
\item Different abstractions for different purposes?\\
\begin{itemize}
\item different levels of abstraction level (as above)
\item OR different perspectives (next slide)
\end{itemize}
\end{itemize}
Note - there are many other ways to use abstraction when thinking
about formal language.
}
\end{slide}
%
% slide 10
%
\begin{slide}{}
{\bf Three aspects of language}
\begin{itemize}
\item syntax -- form, shape, structure 
\item semantics -- meaning
\item pragmatics -- use
\end{itemize}
\end{slide}
%
% slide 11 
%
\begin{slide}{}
{\bf Lexical errors in Java}

What's wrong with

\begin{verbatim}
    fruit = 7peach;
\end{verbatim}

An identifier may not begin with a digit

\end{slide}
%
% slide 12
%
\begin{slide}{}
{\bf Lexical errors in Java}

How about

\begin{verbatim}
    fred = _fred_value:
    joan = _joan_value;
\end{verbatim}

statement delimiter is ; not :
\end{slide}
%
% slide 13
%
\begin{slide}{}
{\bf Lexical errors in Java}

Or maybe

\begin{verbatim}
Abstract Class StackD {
  Abstract Object accept(StackVisitorI ask);
}
\end{verbatim}

keywords `abstract', `class' and `object' should contain only lowercase letters
\end{slide}
%
% slide 14
%
\begin{slide}{}
{\bf Lexical errors in Java}

Or even

\begin{verbatim}
    my_counter = 7 * (loop_counter + 5o5);
\end{verbatim}

the letter `o' should not appear in a number
\end{slide}
%
% slide 15
%
\begin{slide}{}
{\small
{\bf Lexical errors in Java}

What elements of the Java programming language were affected by
these lexical errors?
\begin{itemize}
\item identifier
\item delimiter
\item keyword
\item number (unsigned integer)
\end{itemize}

%\vspace{7ex}

What constitutes the lexis of a programming language?

identifiers, delimiters, keywords, numbers (of various kinds),
comments, \ldots
%\vspace{7ex}

What part of the compiler catches lexical errors?

the scanner
%\vspace{7ex}

How do we specify the lexis of a programming language?

We use regular expressions -- we'll come back to this later.

%\vspace{7ex}
}
\end{slide}

%
% slide 16
%
\begin{slide}{}
{\bf Structure of a Compiler}

\vspace{3ex}
\epsfbox{CompStructFun.eps}
\end{slide}
%
% slide 17
%
\begin{slide}{}
{\bf Main Jobs of a Compiler}

{\bf ANALYSIS of the Source Program}
    \begin{itemize}
    \item lexical analysis
    \item syntactic analysis
    \item context checking
    \end{itemize}

{\bf SYNTHESIS of the Target Program}
    \begin{itemize}
    \item generation of intermediate representation
    \item optimisation
    \item code generation
    \end{itemize}

\end{slide}

%
% slide 18
%
\begin{slide}{}
{\bf The Scanner}

The first job a compiler must do is to read the input
character by character, and group the characters into
useful bits called lexemes.  This task is called
lexical analysis.

Here are some Ada lexemes

%{\tt

%\hspace{3em}    for i in NumberofItems loop \\

%\vspace{2ex}

%\hspace{3em}    ins := 42;

%}
\vspace{3ex}
\epsfbox{lexemes1.eps}

There are some nontrivial problems; illustrated by these
FORTRAN lexemes:

%{\tt

%\hspace{3em}    2.E3 \\

%\vspace{2ex}

%\hspace{3em}    2.EQ.I

%}
\vspace{3ex}
\epsfbox{lexemes2.eps}
\end{slide}

%
% slide 8 
%
\begin{slide}{}
{\small
{\bf The Scanner}

The lexemes are represented as tokens -- often integers -- to 
facilitate further processing by the compiler.

Tokens indicate the lexical class of the lexeme -- identifier,
reserved word, delimeter, \ldots \&c. -- and usually include
a reference to the lexeme itself.

A typical encoding uses one or two digits to represent the lexical
class of a token, with other digits acting as indices into tables
where the lexemes themselves are stored.

For example,
\begin{quote}

    ins := 42;

\end{quote}
might be encoded as
\begin{quote}

    0233 0076 0062 0011

\end{quote}
where the lowest order digit indicates the lexical class
of the token, and the other digits index an entry for
the token itself.
}
\end{slide}
%
% slide 10
%
\begin{slide}{}
{\bf Other jobs done by the scanner}

As well as doing its main job -- recognising lexemes
and encoding them as tokens -- the scanner
\begin{itemize}
\item eliminates comments and ``white space'',
\item formats and lists the source program, and
\item processes compiler directives.
\end{itemize}

\end{slide}
%
% slide 11
%
\begin{slide}{}
{\bf The Scanner in Context}

\vspace{3ex}
\epsfbox{Scanner.eps}

\end{slide}
